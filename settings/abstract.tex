\section*{Vorwort}
Die Manipulation von Booleschen Funktionen ist ein wichtiger Bestandteil des rechnergestützten Konstruierens von digitalen Schaltungen. Diese Arbeit beschreibt ein Softwarepaket, um Boolesche Funktionen in Form eines reduzierten geordneten Binären Entscheidungsdiagramms effizient zu manipulieren, das auf einer effizienten Implementierung vom ITE-Operator basiert. Dazu wird eine Hashtabelle benutzt, um eine kanonische Form des binären Entscheidungsdiagramms zu gewährleisten. Die Verschmelzung dieses Datentyps mit Knoten in eine hybride Datenstruktur führt dazu, dass die Speichernutzung erheblich verbessert wird. Weiterhin erfolgt die Implementierung eines hashbasierten Caches für den ITE-Operator, um weiterhin die Speichernutzung zu verringern. Darüber hinaus wird der Cache insbesondere dadurch verbessert, dass spezielle Regeln für komplementäre Kanten zum Einsatz kommen, wodurch äquivalente Funktionen erkannt werden können. Für die Regulierung der Speichernutzung gibt es eine automatische Speicherbereinigung mit geringen Kosten, um dem Hauptproblem des Speicherplatzes bezüglich binärer Entscheidungsdiagramme entgegenzuwirken. Abschließend demonstrieren experimentelle Ergebnisse die Korrektheit dieses Paketes und werden für einen Vergleich mit einem -- dem neuesten Stand der Technik entsprechenden -- Paket herangezogen.\\\\
Prof. Dr. Rolf Drechsler danke ich für viele hilfreiche Gespräche und Diskussionen sowie die umfassende Betreuung. Weiterhin möchte ich mich bei M. Sc. Marcel Walter und Dipl.-Inf. Kenneth Schmitz für ihr Engagement in Form von intensiven Gesprächen, konstruktiver Kritik und Verbesserungsvorschlägen zu meinen Erkenntnissen bedanken. An dieser Stelle möchte ich auch an Dr. Sabine Kuske ein Danke für ihr Feedback zu dieser Arbeit ausrichten.\\\\
Mein innigster Dank gilt meiner Verwandtschaft, die immer für mich da ist und die Erreichung meiner Ziele zu jeder Zeit unterstützen.\\
Diese Arbeit widme ich euch in Liebe.